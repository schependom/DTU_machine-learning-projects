\documentclass[dtu]{dtuarticle}
\usepackage{parskip} % use enters instead of indents

\newcommand{\todo}[1]{\color{red}[TODO: #1]\color{black}}
\newcommand*\chem[1]{\ensuremath{\mathrm{#1}}}
\usepackage{amsmath}
\usepackage{bm} % bold ITALIC math (for vectors!)
\usepackage{siunitx}
\sisetup{
	per-mode=fraction,
	detect-weight = true,
	detect-family = true,
	separate-uncertainty=true, % Dit is voor de plus-minus
	output-decimal-marker={.}
}

\usepackage{subcaption}

\title{Machine Learning Project 2}
\subtitle{Supervised Learning: Classification and Regression}
\author{Group 94}
\course{02452 Machine Learning}
\address{
	DTU Compute \\
	Fall 2025
}
\date{\today}


\begin{document}

	\maketitle

	\begin{table}[h!]
		\renewcommand{\arraystretch}{1.2}
		\centering
		\begin{subtable}{\textwidth}
			\centering
			\begin{tabular}{l | l}
				\textbf{Name}                 & \textbf{Student number} \\ \hline\hline
				Vincent Van Schependom        & s251739                 \\ \hline
				Diego Armando Mijares Ledezma & s251777                 \\ \hline
				Albert Joe Jensen             & s204601
			\end{tabular}
			\caption{Group members.}
			\label{table:members}
		\end{subtable}
		\\ \vspace*{0.5cm}
		\begin{subtable}{\textwidth}
			\centering
			\begin{tabular}{l | *{3}{|r}}
				\textbf{Task}                 & \textbf{Vincent} & \textbf{Diego} & \textbf{Albert} \\ \hline\hline
				Training \& test loops & 0\%              &            0\% &             0\% \\ \hline
				Coding visualisations         & 0\%              &            0\% &             0\% \\ \hline
				Section 3                     & 0\%              &            0\% &             0\% \\ \hline
				\LaTeX                        & 0\%              &            0\% &             0\%
			\end{tabular}
			\caption{Contributions \& responsabilities table.}
			\label{table:contributions}
		\end{subtable}
		\caption{Group information \& work distribution.}
	\end{table}

	\section*{Introduction}

	The objective of this report is to apply the methods that were discussed during the second
	section of the course \textit{Machine Learning} \cite{book} to a chosen dataset. The aim is to perform
	relevant regression and classification to the data.

	The particular dataset that is being investigated is -- just like in Project 1 -- the \textit{Glass Identification} dataset from 1987 by B. German \cite{dataset}. Table \ref{table:members} lists our full names and student numbers, while Table \ref{table:contributions} shows an overview of the contribution of each team member.

	\tableofcontents

	\newpage

	\section{Regression}

	\todo{Make visualisations of the results in Table \ref{table:e-test-regression}}

	\begin{table}
		\centering
		\renewcommand{\arraystretch}{1.2}
		\begin{tabular}{| r || l | l || l | l || l |}
			\hline
			\textbf{Outer fold} & \multicolumn{2}{c||}{\textbf{ANN}} & \multicolumn{2}{c||}{\textbf{Linear regression}} & \textbf{Baseline}     \\ \hline\hline
			\(i\)               & \(h_i^*\) & \(E_i^{\text{test}}\)  & \(\lambda_i^*\) & \(E_i^{\text{test}}\)          & \(E_i^{\text{test}}\) \\ \hline
1 & 500 & \SI{ 2.607e-02 }{} & \SI{ 0.797 }{} & \SI{ 1.600e-06 }{} & \SI{ 1.970e-05 }{} \\
2 & 480 & \SI{ 1.327e-02 }{} & \SI{ 2.84 }{} & \SI{ 4.249e-07 }{} & \SI{ 1.228e-05 }{} \\
3 & 230 & \SI{ 3.239e-02 }{} & \SI{ 0.325 }{} & \SI{ 4.821e-07 }{} & \SI{ 4.482e-06 }{} \\
4 & 190 & \SI{ 1.592e-02 }{} & \SI{ 0.167 }{} & \SI{ 9.633e-07 }{} & \SI{ 5.047e-06 }{} \\
5 & 120 & \SI{ 1.030e-02 }{} & \SI{ 1.43 }{} & \SI{ 1.500e-06 }{} & \SI{ 3.080e-06 }{} \\
6 & 500 & \SI{ 3.226e-03 }{} & \SI{ 0.01 }{} & \SI{ 2.142e-06 }{} & \SI{ 5.982e-06 }{} \\
7 & 120 & \SI{ 1.603e-02 }{} & \SI{ 0.325 }{} & \SI{ 1.284e-06 }{} & \SI{ 8.186e-06 }{} \\
8 & 240 & \SI{ 3.639e-02 }{} & \SI{ 0.01 }{} & \SI{ 1.018e-06 }{} & \SI{ 7.210e-06 }{} \\
9 & 290 & \SI{ 6.221e-02 }{} & \SI{ 1.27 }{} & \SI{ 1.637e-07 }{} & \SI{ 1.946e-05 }{} \\
10 & 500 & \SI{ 2.641e-02 }{} & \SI{ 3 }{} & \SI{ 5.225e-07 }{} & \SI{ 7.245e-06 }{} \\
			\hline
		\end{tabular}
		\caption{Summary of two-level cross validation for predicting \todo{see Python script}. Hyperparameters and test errors $E_i^\text{test}$ on $\mathcal{D}_i^\text{test}$ per outer fold $i$, for each of the three considered models.}
		\label{table:e-test-regression}
	\end{table}

	\subsection{Linear regression}

	\subsubsection{Aim}

	\todo{Explain what variable is predicted based on which other variables and what you hope to
		accomplish by the regression. Mention your feature transformation choices such as one-of-
		K coding. Since we will use regularization momentarily, apply a feature transformation to
		your data matrix X such that each column has mean 0 and standard deviation 1.}

	We predict the refractive index (RI) of glass samples from their chemical composition: the oxide
	components Na, Mg, Al, Si, K, Ca, Ba, and Fe, and the categorical glass Type. Predicting RI is formulated
	as a regression problem where we evaluate the performance of a regularized linear model (Ridge),
	an artificial neural network (ANN), and a trivial baseline predictor (the mean RI of the training set).
	Numerical features are standardized and categorical ones are encoded appropriately inside the model
	preprocessing pipeline to avoid leakage during cross-validation.


	\subsubsection{Regularization}

	\todo{Introduce a regularization parameter $\lambda$ as discussed in 14 of the lecture notes, and estimate
		the generalization error for different values of $\lambda$. Specifically, choose a reasonable range of
		values of $\lambda$ (ideally one where the generalization error first drop and then increases), and
		for each value use $K = 10$ fold cross-validation (algorithm 5) to estimate the generalization
		error. Include a figure of the estimated generalization error as a function of $\lambda$ in the report
		and briefly discuss the result.}

	\todo{Figure}

	\todo{Reference figure}

	\subsubsection*{Regularization (Ridge Regression)}

	To investigate the effect of regularization, we evaluated Ridge regression models
	across a wide range of $\lambda$ values from $10^{-2}$ to $10^{5}$ using nested
	cross-validation. Figure~\ref{fig:rlr_mse_vs_lambda} shows the average mean squared
	error (MSE) as a function of $\lambda$ on a logarithmic scale.

	The minimum average validation error was found at
	$\hat{\lambda} = 0.2683$ with an average MSE of $1.0033\times10^{-6}$.
	We observe that the curve of MSE is relatively flat around the optimum, indicating
	a stable solution and that the regularization term does not heavily penalize the fit.
	As $\lambda$ increases above $10^2$, the model underfits slightly, leading to a
	gradual rise in error. Conversely, smaller $\lambda$ values ($<10^{-2}$) result
	in marginally higher MSE due to overfitting effects.

	\begin{figure}[ht]
		\centering
		\includegraphics[width=0.75\textwidth]{figures/rlr_mse_vs_lambda.pdf}
		\caption{Mean squared error (MSE) versus regularization strength $\lambda$
		for Ridge regression using nested cross-validation. The optimal $\lambda$
		($\hat{\lambda}=0.2683$) achieved a minimum average MSE of $1.0033\times10^{-6}$.}
		\label{fig:rlr_mse_vs_lambda}
	\end{figure}






	\subsubsection{Model interpretation}

	\todo{Explain how the output, $y$, of the linear model with the lowest generalization error (as
		determined in the previous question) is computed for a given input $x$. What is the effect
		of an individual attribute in $x$ on the output, $y$, of the linear model? Does the effect of
		individual attributes make sense based on your understanding of the problem?}

	\todo{Final equation}

	\todo{...}

	The Ridge regression model was retrained on the entire dataset using the optimal
	regularization parameter $\hat{\lambda}=0.2683$. The resulting coefficients are
	shown below.

	\begin{verbatim}
	Intercept: 1.518365421
	Na: -0.0002574659853
	Mg: 8.28374029e-05
	Al: -0.0008566003247
	Si: -0.001230770752
	K: -0.0002729740513
	Ca: 0.001898817879
	Ba: 0.0006016976661
	Fe: -1.922138819e-05
	\end{verbatim}

	These coefficients reveal the relative contribution of each oxide concentration
	to the refractive index (RI). The elements \textbf{Ca} and \textbf{Ba} exhibit the
	largest positive coefficients, implying that increasing these oxides raises RI.
	Conversely, \textbf{Si}, \textbf{Al}, and \textbf{Na} have negative coefficients,
	indicating that they decrease RI. The small magnitude of coefficients for
	\textbf{Mg} and \textbf{Fe} suggests a weaker influence on RI.

	Overall, the coefficients are consistent with physical expectations from glass
	composition, confirming that Ridge regularization yields a stable and interpretable
	linear model.


	\subsection{Verification of the Ridge model}


	To verify the correctness and predictive performance of the final Ridge model,
	we evaluated it on the full dataset using the optimal $\hat{\lambda}$.
	The resulting performance metrics were:

	\[
	\text{MSE} = 9.999\times10^{-7}, \quad R^2 = 0.8911
	\]

	Figure~\ref{fig:rlr_pred_vs_actual} shows predicted versus actual RI values.
	The points lie closely along the identity line, confirming accurate predictions.
	The residuals in Figure~\ref{fig:rlr_residuals} are randomly distributed around
	zero, suggesting no systematic bias or heteroscedasticity.

	\begin{figure}[ht]
	\centering
	\includegraphics[width=0.45\textwidth]{figures/rlr_predicted_vs_actual.pdf}
	\includegraphics[width=0.45\textwidth]{figures/rlr_residuals_vs_predicted.pdf}
	\caption{Verification of the regularized Ridge regression model.
	Left: predicted versus actual refractive index (RI).
	Right: residuals versus predicted RI. The model shows a high $R^2$ and
	randomly distributed residuals, confirming stable behavior.}
	\label{fig:rlr_pred_vs_actual}
	\label{fig:rlr_residuals}
	\end{figure}




	\subsection{Regularized linear regression vs an Artificial Neural Network}

	\todo{Rewrite this (concisely!) so that it isn't an exact copy of the assignment}

	This section compares regularized linear regression, an artificial neural network, and a baseline model using two-level cross-validation to determine relative performance.

	\todo{Create the table as in the assignment (Vincent) (hola my name is diego and i would like to say bon appetite)}

	\todo{Write the accompanying text on how we retreived the data in the table.}

	\todo{Write out the statistical comparisons using data from the table.}

	\todo{TABLE: Include p-values and confidence intervals for the three pairwise tests in your report.}

	\todo{Conclude on the results from the values in the table and reference the table.}

	The nested cross-validation results for the regression tasks compared three models:
	an artificial neural network (ANN), the regularized Ridge regression (RLR),
	and a baseline mean predictor. Paired t-tests were conducted across folds to assess
	whether performance differences were statistically significant.

	The results are summarized in Table~\ref{tab:reg_pairwise}.
	All p-values below 0.05 indicate statistically significant differences.

	\begin{table}
\caption{Pairwise tests for regression}
\label{tab:reg_pairwise}
\begin{tabular}{lrrrrrr}
\toprule
comparison & mean_diff & sd & t & p & ci_low & ci_high \\
\midrule
ANN vs RLR & 0.0242204 & 0.0168552 & 4.54411 & 0.00139767 & 0.012163 & 0.0362779 \\
ANN vs Baseline & 0.0242122 & 0.0168514 & 4.54356 & 0.00139876 & 0.0121574 & 0.0362669 \\
RLR vs Baseline & -8.25801e-06 & 6.14533e-06 & -4.24943 & 0.00214427 & -1.26541e-05 & -3.86191e-06 \\
\bottomrule
\end{tabular}
\end{table}


	From the results:
	\begin{itemize}
		\item \textbf{ANN vs RLR:} The ANN achieved a lower average test error, with a mean
		difference of $0.0242 \pm 0.0169$ ($t=4.54$, $p=0.0014$). This indicates the ANN
		significantly outperforms Ridge regression on the RI prediction task.
		\item \textbf{ANN vs Baseline:} The ANN also significantly outperformed the baseline
		($t=4.54$, $p=0.0014$), confirming that it captures non-linearities missed by
		linear models.
		\item \textbf{RLR vs Baseline:} Ridge regression slightly outperformed the baseline
		(mean difference $-8.26\times10^{-6}$, $t=-4.25$, $p=0.0021$), confirming that even
		a regularized linear model provides small but significant improvements over the
		naive predictor.
	\end{itemize}

	The ANN's advantage is thus both statistically and practically significant,
	while the Ridge regression, although simpler, remains interpretable and competitive.

	\subsection*{1.3 Classification Model Comparison}

	For the classification task, we compared a decision tree, a logistic regression model,
	and a baseline classifier. Paired t-tests across outer folds were used to assess
	statistical significance. The results are summarized in
	Table~\ref{tab:cls_pairwise}.

	\begin{table}
\caption{Pairwise tests for classification}
\label{tab:cls_pairwise}
\begin{tabular}{lllllll}
\toprule
comparison & mean_diff & sd & t & p & ci_low & ci_high \\
\midrule
Tree vs LogReg & \num{-0.02814} & \num{0.09527} & \num{-0.934} & \num{0.3747} & \num{-0.09629} & \num{0.04001} \\
Tree vs Baseline & \num{-0.3693} & \num{0.131} & \num{-8.913} & \num{9.246e-06} & \num{-0.463} & \num{-0.2755} \\
LogReg vs Baseline & \num{-0.3411} & \num{0.08089} & \num{-13.34} & \num{3.117e-07} & \num{-0.399} & \num{-0.2833} \\
\bottomrule
\end{tabular}
\end{table}


	Interpretation:
	\begin{itemize}
		\item \textbf{Tree vs LogReg:} The difference in accuracy ($-0.0281$) is not statistically
		significant ($p=0.375$), suggesting both models perform similarly.
		\item \textbf{Tree vs Baseline:} The decision tree significantly outperformed
		the baseline ($t=-8.91$, $p<10^{-5}$), indicating it captures meaningful
		structure in the data.
		\item \textbf{LogReg vs Baseline:} Logistic regression also significantly
		outperformed the baseline ($t=-13.34$, $p=3.1\times10^{-7}$), showing that even
		a simple linear classifier captures predictive signal.
	\end{itemize}

	In summary, both classification models substantially improve upon the baseline,
	and their similar performance suggests that the data can be effectively separated
	by relatively simple decision boundaries.

	\section{Classification}

	\begin{table}
		\centering
		\renewcommand{\arraystretch}{1.2}
		\begin{tabular}{| r || l | l || l | l || l |}
			\hline
			\textbf{Outer fold} & \multicolumn{2}{c||}{\textbf{\todo{Decision Tree?}}} & \multicolumn{2}{c||}{\textbf{Logistic regression}} & \textbf{Baseline}     \\ \hline\hline
			\(i\)               & \(h_i^*\) & \(E_i^{\text{test}}\)                    & \(\lambda_i^*\)  & \(E_i^{\text{test}}\)           & \(E_i^{\text{test}}\) \\ \hline
1 & 3 & \SI{ 0.4091 }{} & \SI{ 0.336 }{} & \SI{ 0.2273 }{} & \SI{ 0.5455 }{} \\
2 & 6 & \SI{ 0.2273 }{} & \SI{ 0.0785 }{} & \SI{ 0.3182 }{} & \SI{ 0.5000 }{} \\
3 & 15 & \SI{ 0.3182 }{} & \SI{ 0.00886 }{} & \SI{ 0.4091 }{} & \SI{ 0.8182 }{} \\
4 & 3 & \SI{ 0.3636 }{} & \SI{ 0.0379 }{} & \SI{ 0.4545 }{} & \SI{ 0.9091 }{} \\
5 & 5 & \SI{ 0.3333 }{} & \SI{ 0.00886 }{} & \SI{ 0.2857 }{} & \SI{ 0.5714 }{} \\
6 & 6 & \SI{ 0.4762 }{} & \SI{ 0.00428 }{} & \SI{ 0.4286 }{} & \SI{ 0.8095 }{} \\
7 & 5 & \SI{ 0.2857 }{} & \SI{ 0.00428 }{} & \SI{ 0.3333 }{} & \SI{ 0.6667 }{} \\
8 & 10 & \SI{ 0.2857 }{} & \SI{ 0.336 }{} & \SI{ 0.3333 }{} & \SI{ 0.7619 }{} \\
9 & 4 & \SI{ 0.3333 }{} & \SI{ 0.00428 }{} & \SI{ 0.3810 }{} & \SI{ 0.6667 }{} \\
10 & 7 & \SI{ 0.3333 }{} & \SI{ 0.695 }{} & \SI{ 0.4762 }{} & \SI{ 0.8095 }{} \\
\hline
		\end{tabular}
		\caption{Summary of two-level cross validation for predicting the glass type. Hyperparameters and test errors $E_i^\text{test}$ on $\mathcal{D}_i^\text{test}$ per outer fold $i$, for each of the three considered classificationmodels. \todo{Change $\lambda$ to $C$?}}
		\label{table:e-test-classification}
	\end{table}

	\todo{Choose method 2: ANN, CT, KNN, NB}

	\todo{Make visualisations of the results in \ref{table:e-test-classification}}

	\subsection{Introduction}

	\todo{Explain which classification problem you have chosen to solve. Is it a multi-class or binary
		classification problem?}
	
	We use the glass dataset to predict the glass \texttt{Type} using the oxide concentrations as
	predictors. This is a multi-class classification problem. We compare Logistic Regression,
	a Classification Tree (Decision Tree), and a trivial baseline that always predicts the most frequent
	class present in the training set. All models were evaluated with nested (two-level) cross-validation.


	\subsection{Logistic regression vs [Decision Tree]}

	\todo{Rewrite the assignment below such that it (consisely!) states what we will do in this section.}
	\todo{Perform a statistical evaluation of your three models similar to the previous section. That
		is, compare the three models pairwise.}
	\todo{TABLE: Include p-values and confidence intervals for the three pairwise tests in your report.}
	\todo{Conclude on the results from the values in the table and reference the table.}

	This section compares Logistic Regression and a Decision Tree classifier against the baseline.
	For each outer fold, we optimized the regularization strength $C$ for Logistic Regression
	and the maximum tree depth $h$ for the Decision Tree using inner cross-validation.
	We then computed the test error $E_i^\text{test}$ on each outer test fold.

	A paired statistical evaluation was performed across folds to assess whether
	differences between model performances were significant. The results are shown in
	Table~\ref{tab:cls_pairwise} and summarized below.

	\begin{table}
\caption{Pairwise tests for classification}
\label{tab:cls_pairwise}
\begin{tabular}{lllllll}
\toprule
comparison & mean_diff & sd & t & p & ci_low & ci_high \\
\midrule
Tree vs LogReg & \num{-0.02814} & \num{0.09527} & \num{-0.934} & \num{0.3747} & \num{-0.09629} & \num{0.04001} \\
Tree vs Baseline & \num{-0.3693} & \num{0.131} & \num{-8.913} & \num{9.246e-06} & \num{-0.463} & \num{-0.2755} \\
LogReg vs Baseline & \num{-0.3411} & \num{0.08089} & \num{-13.34} & \num{3.117e-07} & \num{-0.399} & \num{-0.2833} \\
\bottomrule
\end{tabular}
\end{table}


	\begin{itemize}
		\item \textbf{Tree vs LogReg:} The mean difference in error was $-0.0281 \pm 0.0953$
		($t=-0.93$, $p=0.375$), indicating no statistically significant difference. Both models perform comparably.
		\item \textbf{Tree vs Baseline:} The Decision Tree significantly outperformed the baseline
		with $t=-8.91$ and $p=9.25\times10^{-6}$, showing that even a simple tree captures
		predictive relationships among the oxide features.
		\item \textbf{LogReg vs Baseline:} Logistic Regression also achieved a strong improvement
		over the baseline ($t=-13.34$, $p=3.12\times10^{-7}$), confirming that linear decision
		boundaries explain most of the separability in the dataset.
	\end{itemize}

	Figure~\ref{fig:classification_test_errors_by_fold} visualizes the outer test errors per fold for
	the three models, clearly showing that both Logistic Regression and the Decision Tree
	consistently outperform the baseline classifier.

	\begin{figure}[ht]
		\centering
		\includegraphics[width=0.75\textwidth]{figures/classification_test_errors_by_fold.pdf}
		\caption{Test errors $E_i^\text{test}$ across outer folds for each classification model.
		Both the Logistic Regression and Decision Tree outperform the baseline in all folds,
		with Logistic Regression achieving slightly lower error on average.}
		\label{fig:classification_test_errors_by_fold}
	\end{figure}

	
	\subsection{Interpretation of the LR model}

	\todo{Train a logistic regression model using a suitable value of $\lambda$ (see previous exercise). Explain
		how the logistic regression model make a prediction. Are the same features deemed relevant
		as for the regression part of the report?}

	A final Logistic Regression model was trained on the full dataset using the selected regularization
	parameter $C^* = 1 / \hat{\lambda}$ from cross-validation.
	The model assigns a weight vector $w_k$ to each class $k$, such that class scores are
	\[
	s_k(x) = w_{k,0} + w_k^\top x,
	\]
	and predicted probabilities are given by the softmax transformation.

	The sign and magnitude of each coefficient indicate how increasing a given oxide concentration
	affects the log-odds of a sample belonging to class $k$. Features such as
	\textbf{Si}, \textbf{Ca}, and \textbf{Ba} were among the strongest predictors, consistent with the
	results from the regression analysis, where these elements also exhibited large coefficients
	in determining refractive index. This agreement suggests that the same chemical
	components governing RI also drive differences between glass types.

	Overall, Logistic Regression achieves competitive classification accuracy with interpretable
	coefficients, while the Decision Tree offers slightly lower performance but more explicit
	decision boundaries.

	\section*{Use of GenAI}

	
	Generative AI (ChatGPT by OpenAI) was used as a support tool during the project.
	Its role was limited to assisting with code debugging, identifying and fixing minor
	implementation errors, and helping to structure the LaTeX report consistently.
	All analytical choices, data processing steps, and result interpretations were made
	by the project members based on the actual model outputs.

	All code was executed and verified independently by the authors, and every numerical
	result and figure included in the report originates from genuine model runs on the
	provided dataset. Generative AI was not used to fabricate data, produce experimental
	results, or replace the authors' understanding or reasoning.

	\bibliography{citations}
	\bibliographystyle{unsrt}

	\vspace*{1cm}
	\appendix

	\LARGE\bfseries Appendix

	\normalsize\normalfont

	\section{Repository and supplementary materials}
	The full notebook, scripts, and generated figures for this project are available in the project repository:
	\begin{quote}
	\url{https://github.com/schependom/DTU\_machine-learning-projects/tree/main}
	\end{quote}
	This repository contains the data-loading and analysis code that produced the tables and figures cited
	above (see the \texttt{figures/} folder for the PDF outputs referenced in the report).
%
%

%	\section{Test}

\end{document}
