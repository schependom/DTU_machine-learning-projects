\documentclass[dtu]{dtuarticle}
\usepackage{parskip} % use enters instead of indents

\newcommand{\todo}[1]{\color{red}[TODO: #1]\color{black}}
\newcommand*\chem[1]{\ensuremath{\mathrm{#1}}}
\usepackage{amsmath}
\usepackage{bm} % bold ITALIC math (for vectors!)
\usepackage{siunitx}
\usepackage{subcaption}

\title{Machine Learning Project 1}
\subtitle{Data: Feature extraction, and visualization}
\author{Group 94}
\course{02452 Machine Learning}
\address{
	DTU Compute \\
	Fall 2025
}
\date{\today}


\begin{document}

	\maketitle

%	\begin{table}[h!]
%		\renewcommand{\arraystretch}{1.1} % make the spacing a bit nicer
%		\centering
%		\begin{tabular}{l | l}
%			\textbf{Name}                 & \textbf{Student number} \\ \hline\hline
%			Vincent Van Schependom        & s251739                 \\ \hline
%			Diego Armando Mijares Ledezma & s251777                 \\ \hline
%			Albert Joe Jensen             & s204601
%		\end{tabular}
%		\caption{Group members.}
%		\label{table:members}
%	\end{table}
%
%	\begin{table}[h!]
%		\renewcommand{\arraystretch}{1.1} % make the spacing a bit nicer
%		\centering
%		\begin{tabular}{l | *{3}{|l}}
%			\textbf{Task} & \textbf{Vincent} & \textbf{Diego} & \textbf{Albert} \\ \hline\hline
%			Section 1     & 30\%             & 30\%           & 40\%            \\ \hline
%			Section 2     & 40\%             & 30\%           & 30\%            \\ \hline
%			Section 3     & 30\%             & 40\%           & 30\%            \\ \hline
%			Section 4     & 30\%             & 30\%           & 40\%			\\ \hline
%			\LaTeX        & 90\%             & 5\%           & 5\%
%		\end{tabular}
%		\caption{Contributions \& responsabilities table.}
%		\label{table:contributions}
%	\end{table}

	\begin{table}[h!]
		\begin{subtable}{.58\textwidth}
			\begin{tabular}{l | l}
				\textbf{Name}                 & \textbf{Student number} \\ \hline\hline
				Vincent Van Schependom        & s251739                 \\ \hline
				Diego Armando Mijares Ledezma & s251777                 \\ \hline
				Albert Joe Jensen             & s204601
			\end{tabular}
			\caption{Group members.}
			\label{table:members}
		\end{subtable}
		\begin{subtable}{.4\textwidth}
			\begin{tabular}{l | *{3}{|l}}
				\textbf{Task} & \textbf{Vincent} & \textbf{Diego} & \textbf{Albert} \\ \hline\hline
				Section 1     & 30\%             & 30\%           & 40\%            \\ \hline
				Section 2     & 40\%             & 30\%           & 30\%            \\ \hline
				Section 3     & 30\%             & 40\%           & 30\%            \\ \hline
				Section 4     & 30\%             & 30\%           & 40\%			\\ \hline
				\LaTeX        & 90\%             & 5\%           & 5\%
			\end{tabular}
			\caption{Contributions \& responsabilities table.}
			\label{table:contributions}
		\end{subtable}
		\caption{Group information \& work distribution.}
	\end{table}

	\section*{Introduction}

	The objective of this report is to apply the methods that were discussed during the first
	section of the course \textit{Machine Learning} \cite{book} to a chosen dataset. The aim is to get
	a basic understanding of the data prior to the further analysis (project report 2).

	The particular dataset that is being investigated is the \textit{Glass Identification} dataset from 1987 by B. German \cite{dataset}. Table \ref{table:members} lists our full names and student numbers, while Table \ref{table:contributions} shows an overview of the contribution of each team member.

	\tableofcontents

	\newpage

	\section{The \textit{Glass Identification} dataset}

	\todo{The introduction to the data set.}

	\section{A close look at the different attributes}

	\begin{table}
		\centering
		%\renewcommand{\arraystretch}{1.3} % make the spacing a bit nicer
		\begin{tabular}{l|l|l}
			\textbf{Attribute} & \textbf{Description}                    & \textbf{Type of variable} \\ \hline \hline
			\texttt{ID}        & Observation ID (excluded from analysis) & Numeric (discrete)        \\ \hline
			\texttt{RI}        & Refractive Index                        & Continuous                \\ \hline
			\texttt{Na}        & Sodium oxide ($\chem{Na_2 O}$)          & Continuous                \\ \hline
			\texttt{Mg}        & Magnesium oxide ($\chem{Mg O}$)         & Continuous                \\ \hline
			\texttt{Al}        & Aluminum oxide ($\chem{Al_2 O_3}$)      & Continuous                \\ \hline
			\texttt{Si}        & Silicon oxide ($\chem{Si O_2}$)         & Continuous                \\ \hline
			\texttt{K}         & Potassium oxide ($\chem{K_2 O}$)        & Continuous                \\ \hline
			\texttt{Ca}        & Calcium oxide ($\chem{Ca O }$)          & Continuous                \\ \hline
			\texttt{Ba}        & Barium oxide ($\chem{Ba O}$)            & Continuous                \\ \hline
			\texttt{Fe}        & Iron oxide ($\chem{Fe_2 O_3}$)          & Continuous                \\ \hline
			\texttt{type}      & Type of glass                           & Nominal                   \\
		\end{tabular}
		\caption{\todo{Fill in.}}
		\label{table:attributes}
	\end{table}

	\begin{table}
		\centering
		%\renewcommand{\arraystretch}{1.3} % make the spacing a bit nicer
		\begin{tabular}{r|l|l}
			\textbf{} & \textbf{Abbreviation in dataset} & \textbf{Description}                 \\ \hline\hline
			        1 & \texttt{BW-FP}                   & Building Window, Float Processed     \\ \hline
			        2 & \texttt{BW-NFP}                  & Building Window, Non Float Processed \\ \hline
			        3 & \texttt{VW-FP}                   & Vehicle Window, Float Processed      \\ \hline
			        4 & \texttt{VW-NFP}                  & Vehicle Window, Non Float Processed  \\ \hline
			        5 & \texttt{containers}              & Containers                           \\ \hline
			        6 & \texttt{tableware}               & Tableware (e.g. \dots)               \\ \hline
			        7 & \texttt{headlamps}               & Headlamps (e.g. \dots)
		\end{tabular}
		\caption{\todo{Fill in.}}
		\label{table:types}
	\end{table}

	\todo{Detailed explaination of the attributes of the data.}

	\section{Descriptive analysis of the dataset}

	\todo{...}

	\subsection{Extreme values and outliers}

	\textbf{Note that the IQR `rule' is not a ground truth, but rather a way to detect \textit{possible} outliers.}

	\todo{...}

	\subsection{Distribution of the attributes}
	\label{section:distribution}

	\textbf{Note that there are no non-float processed vehicle windows! (\#\texttt{VW-NFP}=0)}

	\todo{...}

	\begin{table}[h!]
		\begin{subtable}{0.6\textwidth}
			\begin{tabular}{r | *{6}{| r}}
				            & \textbf{Mean} & \textbf{Std.} & \textbf{Min} & \textbf{Max} & \textbf{Skew} & \textbf{Kurtosis} \\ \hline\hline
				\texttt{RI} &               &               &              &              &               &                   \\ \hline
				\texttt{Na} &               &               &              &              &               &                   \\ \hline
				\texttt{Mg} &               &               &              &              &               &                   \\ \hline
				\texttt{Al} &               &               &              &              &               &                   \\ \hline
				\texttt{Si} &               &               &              &              &               &                   \\ \hline
				 \texttt{K} &               &               &              &              &               &                   \\ \hline
				\texttt{Ca} &               &               &              &              &               &                   \\ \hline
				\texttt{Ba} &               &               &              &              &               &                   \\ \hline
				\texttt{Fe} &               &               &              &              &               &
			\end{tabular}
			\caption{Summary statistics.}
			\label{table:summary-stats}
		\end{subtable}
		\hspace*{0\textwidth}
		\begin{subtable}{.3\textwidth}
			\begin{tabular}{r|l|l|l}
				  & \texttt{type}       & \textbf{Cnt.} & \textbf{Freq.} \\ \hline\hline
				1 & \texttt{BW-FP}      &                &                \\ \hline
				2 & \texttt{BW-NFP}     &                &                \\ \hline
				3 & \texttt{VW-FP}      &                &                \\ \hline
				4 & \texttt{VW-NFP}     &                &                \\ \hline
				5 & \texttt{containers} &                &                \\ \hline
				6 & \texttt{tableware}  &                &                \\ \hline
				7 & \texttt{headlamps}  &                &
			\end{tabular}
			\caption{Absolute and relative frequencies of \texttt{type}.}
			\label{table:frequencies}
		\end{subtable}
		\caption{\todo{Fill in.}}
	\end{table}

	\subsection{Correlation between attributes}

	\todo{...}

	\section{Principal Component Analysis}

	\todo{Explain why we standardise.}

	Because the attributes are on completely different scales (Section \ref{section:distribution}), we standardise the variables by subtracting their mean and subsequently dividing by the standard deviation.

	\subsection{Dimension reduction}

	We aim to reduce the 9-dimensional dataset into an $M$-dimensional one (with $M < 9$).

	\todo{How many PC's do we keep? I think $M=5$.}

	\subsection{Principal directions}

	The \textit{principal directions} of the (first $M$) principal components are the rotations, corresponding to each principal component $\text{PC}_i = \bm{v_i}$ in the transform-matrix $\bm{V}_M$. This matrix is used when computing the projected coordinates $\bm{B} = \bm{V}_M \bm{X}$ of the original data $\bm{X}$ onto the subspace spanned by the first $M$ principal components.

	\begin{table}[h!]
		\centering
		\begin{tabular}{r | *{3}{|r}}
			\textbf{Variable} & $\textbf{PC}_1$ & ... & $\textbf{PC}_M = \textbf{PC}_{5??}$ \\ \hline \hline
			\texttt{RI} & \num{0} & ... & \num{0} \\
			\texttt{Na} & \num{0} & ... & \num{0} \\
			\texttt{Mg} & \num{0} & ... & \num{0} \\
			\texttt{Al} & \num{0} & ... & \num{0} \\
			\texttt{Si} & \num{0} & ... & \num{0} \\
			\texttt{K} & \num{0} & ... & \num{0} \\
			\texttt{Ca} & \num{0} & ... & \num{0} \\
			\texttt{Ba} & \num{0} & ... & \num{0} \\
			\texttt{Fe} & \num{0} & ... & \num{0}
		\end{tabular}
		\caption{The principal directions (a.k.a. the \textit{loadings}) of the first $M$ principal components $\text{PC}_i = \bm{v_i}$ in the rotation matrix $\bm{V}_M$. \todo{Describe what these directions mean in terms of the original attributes.}}
		\label{table:loadings}
	\end{table}

	\subsection{Projected data}

	\todo{Visualisations of the projected data}

	\section*{Summary}

	\todo{A short summary of what we discussed in the whole paper.}

	\bibliography{citations}
	\bibliographystyle{unsrt}

\end{document}
