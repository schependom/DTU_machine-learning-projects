\documentclass[dtu]{dtuarticle}
\usepackage{parskip} % use enters instead of indents

\newcommand{\todo}[1]{\color{red}[TODO: #1]\color{black}}

\title{Machine Learning Project 1}
\subtitle{Data: Feature extraction, and visualization}
\author{Group 94}
\course{02452 Machine Learning}
\address{
	DTU Compute \\
	Fall 2025
}
\date{\today}

\begin{document}

	\maketitle

	\begin{table}[h!]
		\renewcommand{\arraystretch}{1.3} % make the spacing a bit nicer
		\centering
		\begin{tabular}{l | l}
			\textbf{Name}                 & \textbf{Student number} \\ \hline\hline
			Vincent Van Schependom        & s251739                 \\ \hline
			Diego Armando Mijares Ledezma & s251777                 \\ \hline
			Albert Joe Jensen             & s204601
		\end{tabular}
		\caption{Group members.}
		\label{table:members}
	\end{table}

	\begin{table}[h!]
		\renewcommand{\arraystretch}{1.3} % make the spacing a bit nicer
		\centering
		\begin{tabular}{l | *{3}{|l}}
			\textbf{Task} & \textbf{Vincent} & \textbf{Diego} & \textbf{Albert} \\ \hline\hline
			Section 1     & 30\%             & 30\%           & 40\%            \\ \hline
			Section 2     & 40\%             & 30\%           & 30\%            \\ \hline
			Section 3     & 30\%             & 40\%           & 30\%            \\ \hline
			Section 4     & 30\%             & 30\%           & 40\%			\\ \hline
			\LaTeX        & 80\%             & 10\%           & 10\%
		\end{tabular}
		\caption{Contributions \& responsabilities table.}
		\label{table:contributions}
	\end{table}

	\section*{Introduction}

	The objective of this report is to apply the methods that were discussed during the first
	section of the course \textit{Machine Learning} \cite{book} to a chosen dataset. The aim is to get
	a basic understanding of the data prior to the further analysis (project report 2).

	The particular dataset that is being investigated is the \textit{Glass Identification} dataset from 1987 by B. German \cite{dataset}.

	Table \ref{table:members} lists our full names and student numbers, while Table \ref{table:contributions} shows an overview of the contribution of each team member.

	\tableofcontents

	\newpage

	\section{The \textit{Glass Identification} dataset}

	\todo{The introduction to the data set.}

	\section{A close look at the different attributes}

	\todo{Detailed explaination of the attributes of the data.}

	\section{Descriptive analysis of the dataset}

	\todo{...}

	\subsection{Extreme values and outliers}

	\todo{...}

	\subsection{Distribution of the attributes}

	\todo{...}

	\subsection{Correlation between attributes}

	\todo{...}

	\section{Principal Component Analysis}

	\todo{...}

	\section*{Summary}

	\todo{A short summary of what we discussed in the whole paper.}

	\bibliography{citations}
	\bibliographystyle{unsrt}

\end{document}
